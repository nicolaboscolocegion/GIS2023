\section{Starting Situation}
\label{sec:starting_situation}

In order to provide a comprehensive understanding of the project context, it is important to analyze the starting situation, including the existing technology, available data, and market analysis.

\subsection{Technology} The Province of Belluno currently operates an IT sector responsible for managing their Management Information System (MIS) using commercial software and the PostgreSQL database management system (DBMS). 
                        
                        However, a specific Geographic Information System (GIS) application tailored for hydrography management is not currently in place: our task is to develop a customized GIS solution to address this gap and fulfill the province's requirements.

\subsection{Exsisting Data} The region Veneto already possesses a topographic database built in accordance with the Italian national standard at the NC5 level. This database provides detailed information about the province's terrain and water features. The data is stored in the ETRF2000 reference system and can be accessed through OGC WMS (Web Map Service) and WFS (Web Feature Service) non-transactional services. 
                            
Additionally, orthophotos at a scale of 1:5000, also based on the ETRF2000 reference system, are available for visualization through WMS.

\subsection{Market Analysis} Extensive research indicates that there are currently no off-the-shelf or packaged solutions available in the market that adequately address the requirements of the hydrography management system for the Province of Belluno. 

                            This market gap presents an exciting opportunity for our team to develop a tailored solution that precisely meets the province's needs. By creating a custom application, we can ensure the fulfillment of all functional and nonfunctional requirements while complying with the relevant standards and regulations governing water resource management.

\subsection{Conclusions} Analyzing the starting situation in terms of existing technology, available data, and market conditions, it is evident that the development of a dedicated hydrography management system is crucial to effectively manage the province's water resources and address the current market gap.